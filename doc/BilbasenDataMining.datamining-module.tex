%
% API Documentation for API Documentation
% Module BilbasenDataMining.datamining
%
% Generated by epydoc 3.0.1
% [Thu Dec  4 13:59:41 2014]
%

%%%%%%%%%%%%%%%%%%%%%%%%%%%%%%%%%%%%%%%%%%%%%%%%%%%%%%%%%%%%%%%%%%%%%%%%%%%
%%                          Module Description                           %%
%%%%%%%%%%%%%%%%%%%%%%%%%%%%%%%%%%%%%%%%%%%%%%%%%%%%%%%%%%%%%%%%%%%%%%%%%%%

    \index{BilbasenDataMining \textit{(package)}!BilbasenDataMining.datamining \textit{(module)}|(}
\section{Module BilbasenDataMining.datamining}

    \label{BilbasenDataMining:datamining}
Datamining module

This module contains all the data mining methods.


%%%%%%%%%%%%%%%%%%%%%%%%%%%%%%%%%%%%%%%%%%%%%%%%%%%%%%%%%%%%%%%%%%%%%%%%%%%
%%                               Functions                               %%
%%%%%%%%%%%%%%%%%%%%%%%%%%%%%%%%%%%%%%%%%%%%%%%%%%%%%%%%%%%%%%%%%%%%%%%%%%%

  \subsection{Functions}

    \label{BilbasenDataMining:datamining:get_distribution_all_brands}
    \index{BilbasenDataMining \textit{(package)}!BilbasenDataMining.datamining \textit{(module)}!BilbasenDataMining.datamining.get\_distribution\_all\_brands \textit{(function)}}

    \vspace{0.5ex}

\hspace{.8\funcindent}\begin{boxedminipage}{\funcwidth}

    \raggedright \textbf{get\_distribution\_all\_brands}(\textit{tablename})

    \vspace{-1.5ex}

    \rule{\textwidth}{0.5\fboxrule}
\setlength{\parskip}{2ex}
    Calculates a frequency distribution of all car brands in the given 
    table and returns a panda dataframe of the distribution with 
    brand-names in the first column, the number a brand appears in the 
    table in the second column and the corresponding percentage in the last
    column

\setlength{\parskip}{1ex}
    \end{boxedminipage}

    \label{BilbasenDataMining:datamining:get_distribution_one_brand}
    \index{BilbasenDataMining \textit{(package)}!BilbasenDataMining.datamining \textit{(module)}!BilbasenDataMining.datamining.get\_distribution\_one\_brand \textit{(function)}}

    \vspace{0.5ex}

\hspace{.8\funcindent}\begin{boxedminipage}{\funcwidth}

    \raggedright \textbf{get\_distribution\_one\_brand}(\textit{tablename}, \textit{brand})

    \vspace{-1.5ex}

    \rule{\textwidth}{0.5\fboxrule}
\setlength{\parskip}{2ex}
    Calculates a frequency distribution of a particular car brand in the 
    given table in terms of the different models for that brand. Returns a 
    panda dataframe of the distribution with model-names in the first 
    column, the number a model appears in the table in the second column, 
    and the corresponding percentage in the last column

\setlength{\parskip}{1ex}
    \end{boxedminipage}

    \label{BilbasenDataMining:datamining:create_distribution}
    \index{BilbasenDataMining \textit{(package)}!BilbasenDataMining.datamining \textit{(module)}!BilbasenDataMining.datamining.create\_distribution \textit{(function)}}

    \vspace{0.5ex}

\hspace{.8\funcindent}\begin{boxedminipage}{\funcwidth}

    \raggedright \textbf{create\_distribution}(\textit{series}, \textit{columns}, \textit{n\_models})

    \vspace{-1.5ex}

    \rule{\textwidth}{0.5\fboxrule}
\setlength{\parskip}{2ex}
    Calculates a frequency distribution (using the nltk-package) of a  
    given panda series object. Returns a panda dataframe object containing 
    the frequency distribution with the values of the given series as the 
    first column, the number of time a certain value appear in the given 
    series in     the second column, and the corresponding percentage in 
    the third column. The DataFrame object returned will have labelled 
    columns according to the given columns list.

\setlength{\parskip}{1ex}
    \end{boxedminipage}

    \label{BilbasenDataMining:datamining:get_location_distribution_all_brands}
    \index{BilbasenDataMining \textit{(package)}!BilbasenDataMining.datamining \textit{(module)}!BilbasenDataMining.datamining.get\_location\_distribution\_all\_brands \textit{(function)}}

    \vspace{0.5ex}

\hspace{.8\funcindent}\begin{boxedminipage}{\funcwidth}

    \raggedright \textbf{get\_location\_distribution\_all\_brands}(\textit{table})

    \vspace{-1.5ex}

    \rule{\textwidth}{0.5\fboxrule}
\setlength{\parskip}{2ex}
    Calculates the distribution of all brands based on sale locations in 
    the specified table. A pandas Series with the distribution and the 
    locations as index is returned.

\setlength{\parskip}{1ex}
    \end{boxedminipage}

    \label{BilbasenDataMining:datamining:get_location_distribution_one_brand}
    \index{BilbasenDataMining \textit{(package)}!BilbasenDataMining.datamining \textit{(module)}!BilbasenDataMining.datamining.get\_location\_distribution\_one\_brand \textit{(function)}}

    \vspace{0.5ex}

\hspace{.8\funcindent}\begin{boxedminipage}{\funcwidth}

    \raggedright \textbf{get\_location\_distribution\_one\_brand}(\textit{table}, \textit{brand})

    \vspace{-1.5ex}

    \rule{\textwidth}{0.5\fboxrule}
\setlength{\parskip}{2ex}
    Calculates the distribution of the specified brand based on sale 
    locations in the specified table. A pandas Series with the distribution
    and the locations as index is returned.

\setlength{\parskip}{1ex}
    \end{boxedminipage}

    \label{BilbasenDataMining:datamining:extract_brands}
    \index{BilbasenDataMining \textit{(package)}!BilbasenDataMining.datamining \textit{(module)}!BilbasenDataMining.datamining.extract\_brands \textit{(function)}}

    \vspace{0.5ex}

\hspace{.8\funcindent}\begin{boxedminipage}{\funcwidth}

    \raggedright \textbf{extract\_brands}(\textit{models})

    \vspace{-1.5ex}

    \rule{\textwidth}{0.5\fboxrule}
\setlength{\parskip}{2ex}
    Extracts and returns the exact brand names from the given list of 
    models, i.e. from the list ["Audi A8", "Mercedes SLK"], the list 
    ["Audi", "Mercedes"] will be extracted and returned

\setlength{\parskip}{1ex}
    \end{boxedminipage}

    \label{BilbasenDataMining:datamining:simplify_model_names}
    \index{BilbasenDataMining \textit{(package)}!BilbasenDataMining.datamining \textit{(module)}!BilbasenDataMining.datamining.simplify\_model\_names \textit{(function)}}

    \vspace{0.5ex}

\hspace{.8\funcindent}\begin{boxedminipage}{\funcwidth}

    \raggedright \textbf{simplify\_model\_names}(\textit{models})

    \vspace{-1.5ex}

    \rule{\textwidth}{0.5\fboxrule}
\setlength{\parskip}{2ex}
    Simplifies the car model names in the given list of models, i.e. 
    removes irrelevant information in the names, number of doors ('4d'), 
    size of engine ('2.0L') etc.

\setlength{\parskip}{1ex}
    \end{boxedminipage}

    \label{BilbasenDataMining:datamining:get_fastest_cars}
    \index{BilbasenDataMining \textit{(package)}!BilbasenDataMining.datamining \textit{(module)}!BilbasenDataMining.datamining.get\_fastest\_cars \textit{(function)}}

    \vspace{0.5ex}

\hspace{.8\funcindent}\begin{boxedminipage}{\funcwidth}

    \raggedright \textbf{get\_fastest\_cars}(\textit{table})

    \vspace{-1.5ex}

    \rule{\textwidth}{0.5\fboxrule}
\setlength{\parskip}{2ex}
    Finds the fastest car(s) in the given table (i.e. the car(s) that goes 
    fastest from 0-100 km/t). The result is returned as a pandas DataFrame 
    containing the entire database row of that car.

\setlength{\parskip}{1ex}
    \end{boxedminipage}

    \label{BilbasenDataMining:datamining:get_cheapest_cars}
    \index{BilbasenDataMining \textit{(package)}!BilbasenDataMining.datamining \textit{(module)}!BilbasenDataMining.datamining.get\_cheapest\_cars \textit{(function)}}

    \vspace{0.5ex}

\hspace{.8\funcindent}\begin{boxedminipage}{\funcwidth}

    \raggedright \textbf{get\_cheapest\_cars}(\textit{table})

    \vspace{-1.5ex}

    \rule{\textwidth}{0.5\fboxrule}
\setlength{\parskip}{2ex}
    Finds the cheapest car(s) in the given table, that is not 0DKK and is 
    not a leasing car. The result is returned as a pandas DataFrame 
    containing the entire database row of that car.

\setlength{\parskip}{1ex}
    \end{boxedminipage}

    \label{BilbasenDataMining:datamining:get_most_expensive_cars}
    \index{BilbasenDataMining \textit{(package)}!BilbasenDataMining.datamining \textit{(module)}!BilbasenDataMining.datamining.get\_most\_expensive\_cars \textit{(function)}}

    \vspace{0.5ex}

\hspace{.8\funcindent}\begin{boxedminipage}{\funcwidth}

    \raggedright \textbf{get\_most\_expensive\_cars}(\textit{table})

    \vspace{-1.5ex}

    \rule{\textwidth}{0.5\fboxrule}
\setlength{\parskip}{2ex}
    Finds the most expensive car(s) in the given table. The result is 
    returned as a pandas DataFrame containing the entire database row of 
    that car.

\setlength{\parskip}{1ex}
    \end{boxedminipage}

    \label{BilbasenDataMining:datamining:get_most_ecofriendly_cars}
    \index{BilbasenDataMining \textit{(package)}!BilbasenDataMining.datamining \textit{(module)}!BilbasenDataMining.datamining.get\_most\_ecofriendly\_cars \textit{(function)}}

    \vspace{0.5ex}

\hspace{.8\funcindent}\begin{boxedminipage}{\funcwidth}

    \raggedright \textbf{get\_most\_ecofriendly\_cars}(\textit{table})

    \vspace{-1.5ex}

    \rule{\textwidth}{0.5\fboxrule}
\setlength{\parskip}{2ex}
    Finds the most eco friendly car(s) in the given table, i.e. the car(s) 
    that goes most KMs on a liter petrol. The result is returned as a 
    pandas DataFrame containing the entire database row of that car.

\setlength{\parskip}{1ex}
    \end{boxedminipage}

    \label{BilbasenDataMining:datamining:analyze_description}
    \index{BilbasenDataMining \textit{(package)}!BilbasenDataMining.datamining \textit{(module)}!BilbasenDataMining.datamining.analyze\_description \textit{(function)}}

    \vspace{0.5ex}

\hspace{.8\funcindent}\begin{boxedminipage}{\funcwidth}

    \raggedright \textbf{analyze\_description}(\textit{description})

    \vspace{-1.5ex}

    \rule{\textwidth}{0.5\fboxrule}
\setlength{\parskip}{2ex}
    A very simple sentiment analysis function Assigns a score to a 
    sentence, based on how many positive or negative words appear in that 
    sentence. If there are no negative or positive words,     the final 
    score will be 0. If there are more positive than negative words, the 
    score will be positive - how positive will depend on how many positive 
    / negative words. As it it most likely that there will be many positive
    words compared to negative words, a negative word is counted twice, as 
    the score should reflect a clear difference between a car description 
    containing only positive words, and a car description containing words 
    like "buler" and "dårligt", which clearly is a worse car.

\setlength{\parskip}{1ex}
    \end{boxedminipage}

    \label{BilbasenDataMining:datamining:calculate_best_offer}
    \index{BilbasenDataMining \textit{(package)}!BilbasenDataMining.datamining \textit{(module)}!BilbasenDataMining.datamining.calculate\_best\_offer \textit{(function)}}

    \vspace{0.5ex}

\hspace{.8\funcindent}\begin{boxedminipage}{\funcwidth}

    \raggedright \textbf{calculate\_best\_offer}(\textit{table}, \textit{model})

    \vspace{-1.5ex}

    \rule{\textwidth}{0.5\fboxrule}
\setlength{\parskip}{2ex}
    This method calculates the best offer for a given car model. The 
    calculation is based on creating a linear regression model of the 
    attributes: car description, mileage and age, with prices as y-values. 
    The description of the car is analyzed to get a rank value, using the 
    modules analyze\_description method. Based on the linear regression 
    model, a prediction of the prices is calculated, and the differences 
    between the predicted prices and the actual prices are calculated to 
    find the biggest difference (the largest negative value), which is the 
    best offer. The returned result is a pandas DataFrame along with the 
    difference.

\setlength{\parskip}{1ex}
    \end{boxedminipage}


%%%%%%%%%%%%%%%%%%%%%%%%%%%%%%%%%%%%%%%%%%%%%%%%%%%%%%%%%%%%%%%%%%%%%%%%%%%
%%                               Variables                               %%
%%%%%%%%%%%%%%%%%%%%%%%%%%%%%%%%%%%%%%%%%%%%%%%%%%%%%%%%%%%%%%%%%%%%%%%%%%%

  \subsection{Variables}

    \vspace{-1cm}
\hspace{\varindent}\begin{longtable}{|p{\varnamewidth}|p{\vardescrwidth}|l}
\cline{1-2}
\cline{1-2} \centering \textbf{Name} & \centering \textbf{Description}& \\
\cline{1-2}
\endhead\cline{1-2}\multicolumn{3}{r}{\small\textit{continued on next page}}\\\endfoot\cline{1-2}
\endlastfoot\raggedright \_\-\_\-p\-a\-c\-k\-a\-g\-e\-\_\-\_\- & \raggedright \textbf{Value:} 
{\tt \texttt{'}\texttt{BilbasenDataMining}\texttt{'}}&\\
\cline{1-2}
\end{longtable}

    \index{BilbasenDataMining \textit{(package)}!BilbasenDataMining.datamining \textit{(module)}|)}
